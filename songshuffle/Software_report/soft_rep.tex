\documentclass{article}
\usepackage{listings}

\begin{document}

\title{Software Assignment Report}
\author{Rithika Gajarla (BT22BTECH11013)}

\maketitle

The code is used to play the songs using the \texttt{pygame} library. By shuffling the audio files, the songs are played in a random order.
\section{Libraries imported}
\begin{itemize}\item \textbf{\texttt{os}}\end{itemize}
\begin{itemize}\item \textbf{\texttt{random}}- For shuffling the audio files\end{itemize}
\begin{itemize}\item \textbf{\texttt{pygame}}- For playing the audio files\end{itemize}
  
\section{Functions used}   
        \begin{itemize}
           
        
                \item The list of audio files is shuffled randomly using \texttt{random.shuffle()}.
                \item It then iterates over each audio file in the shuffled list, constructs the full file path using \texttt{os.path.join()}, and calls the \texttt{play\_song()} function to play the audio file.
               
            \end{itemize}
       \begin{itemize}   
            \item \texttt{play\_song()}: This function performs the following steps:
            
                \item It initializes the \texttt{pygame.mixer} module for audio playback using \texttt{pygame.mixer.init()}.
                \item The specified song is loaded using \texttt{pygame.mixer.music.load()}.
                \item The song is played using \texttt{pygame.mixer.music.play()}.
                \item It enters a loop that waits for the song to finish playing by continuously checking \texttt{pygame.mixer.music.get\_busy()}.
            \end{itemize}
\begin{itemize}  
    \item \textbf{Usage Example}: In this section, the code sets the \texttt{folder\_path} variable to \texttt{"./songs"}, indicating that it will look for songs in the "songs" subdirectory of the current working directory.
     \end{itemize}
     
\section{Link to the code}  
     https://github.com/Rithika83/Probability/blob/main/songshuffle/playlist.py


\end{document}

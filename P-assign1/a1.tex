\documentclass[12pt,twocolumn,notitlepage]{article}
\usepackage[margin=0.5in]{geometry}
\usepackage{amsmath}
\usepackage{gensymb}
\usepackage{graphicx}
\usepackage{amsthm}
\usepackage{mathrsfs}
\usepackage{txfonts}
\usepackage{cite}
\usepackage{cases}
\usepackage{subfig}
\usepackage[breaklinks=true]{hyperref}
\usepackage{listings}
\usepackage[latin1]{inputenc}
\usepackage{color}
\usepackage{array}
\usepackage{longtable}
\usepackage{calc}
\usepackage{multirow}
\usepackage{hhline}
\usepackage{ifthen}
\usepackage{amssymb}
\providecommand{\pr}[1]{\ensuremath{\Pr\left(#1\right)}}
\providecommand{\sbrak}[1]{\ensuremath{{}\left[#1\right]}}
\providecommand{\lsbrak}[1]{\ensuremath{{}\left[#1\right.}}
\providecommand{\rsbrak}[1]{\ensuremath{{}\left.#1\right]}}
\providecommand{\brak}[1]{\ensuremath{\left(#1\right)}}
\providecommand{\lbrak}[1]{\ensuremath{\left(#1\right.}}
\providecommand{\rbrak}[1]{\ensuremath{\left.#1\right)}}
\providecommand{\cbrak}[1]{\ensuremath{\left\{#1\right\}}}
\providecommand{\lcbrak}[1]{\ensuremath{\left\{#1\right.}}
\providecommand{\rcbrak}[1]{\ensuremath{\left.#1\right\}}}
\newcommand*{\comb}[2]{{}^{#1}C_{#2}}
\title{Probability Assignment 1 (10.15.1.23)}
\author{Gajarla Rithika (BT22BTECH11013)}
\date{}
\begin{document}
\maketitle
\textbf{Question}
A game consists of tossing a one rupee coin 3 times and noting its outcome each time. Hanif wins if all the tosses give the same result i.e., three heads or three tails, and loses otherwise. Calculate the probability that Hanif will lose the game.\\

\textbf{Solution}
The total number of possible outcomes when tossing a coin 3 times is $2^3 = 8$. The total outcomes are
\begin{align}
        [
\{ HHH, HHT, HTH, THH, HTT, THT, TTH, TTT \}
]
\end{align}
Out of these 8 possible outcomes, there are only 2 outcomes where all tosses give the same result: HHH (three heads) and TTT (three tails).
Therefore, the probability of Hanif losing the game is given by:
\begin{equation*}
    P(A) = \frac{\text{number of required outcomes}}{\text{total number of outcomes}} = \frac{6}{8} = \frac{3}{4}.
\end{equation*}
Hence, the probability that Hanif will lose the game is $\frac{3}{4}$ or 0.75.
 The Table \ref{table:1} shows the declaration of random variable.\\
  \setlength{\tabcolsep}{18pt}
 \renewcommand{\arraystretch}{2.15}
 \begin{table}[hbp] 
\centering
\caption{RANDOM VARIABLE DECLARATION}
\label{table:1}
\begin{tabular}{|c|c|c|}
\hline
Random Variable  & Value of the random variable    & Event                            \\
\hline
\multirow{6}{*}X  & 0                            & number of heads tossed is zero (TTT)\\
         
\cline{2-3}
                 & 1                            & number of heads tossed is one (HTT, THT, TTH)            \\
                 \cline{2-3}
                 & 2                           & number of heads tossed is two  (HHT, HTH, HHT)           \\
                 \cline{2-3}
                 & 3                           & number of heads tossed is three (HHH)\\
                 \cline{2-3}
                
\hline

\end{tabular}

\end{table}

\end{document}
